Notes from Cheng Li, on what he does for clustering:
I basically followed the methods in the attached paper (page 2, upper-right
corner).
  
it's not the standard avarage linkage, instead, after two genes (or nodes)
are merged, the resultant node has expression profile as the avereage the
the two merged ones (after standardization). A description for anther
project using dchip is as follows:
  
Hierarchical clustering analysis (3) is used to group genes with same
expression pattern. A genes is selected for clustering if (1) its expression
values in the 20 samples has coefficient of variation (standard deviation /
mean) between 0.5 to 10 (2) it is called ��Present�� by GeneChip? in more
than 5 samples. Then the expression values for a gene across the 20 samples
are standardized to have mean 0 and standard deviation 1 by linear
transformation, and the distance between two genes is defined as 1 - r where
r is the standard correlation coefficient between the 20 standardize values
of two genes. Two genes with the closest distance are first merged into a
super-gene and connected by branches with length representing their
distance, and are deleted for future merging. The expression level of the
newly formed super-gene is the average of standardized expression levels of
the two genes (average-linkage) for each sample. Then the next pair of genes
(super-genes) with the smallest distance are chosen to merge and the process
is repeated until all genes are merged into one cluster. The dendrogram in
Figure ? illustrates the final clustering tree, where genes close to each
other have high similarity in their standardized expression values across
the 20 samples.
  
  

